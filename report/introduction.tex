In this report, we describe our implementation of a 3D shape retrieval system which aims to return the shapes that are most similar to a user-given example.
Figure \ref{fig:mr-pipeline} presents a pipeline of our system.
The processes coloured in blue are performed offline, whilst the ones coloured in red are performed online (i.e.\ when a query is made).

This report is structured as follows, in Section \ref{section:setting-the-environment} we describe technical details about the environment we use to develop this system such as the dataset and libraries we use.
Section \ref{section:preprocessing} describes how we preprocess the shapes from the dataset.
In Section \ref{section:feature-extraction}, we describe what features we use in order to discriminate the shapes.
The approaches to computing the distance/dissimilarity between the shapes are presented and described in Section \ref{section:discriminating-using-features}.
Two approaches that help to reduce the complexity of the computations, allowing for scalability of our system for bigger data sets, are described in Section \ref{section:scalability}.
Section \ref{section:retrieval-system} brings a brief overview along with instructions on how to use our system from a user perspective.
A quantitative evaluation of our system is presented in Section \ref{section:evaluation} along with discussions regarding how well the system performs for different shapes.
The report ends with Section \ref{section:conclusions}, in which we discuss the strong and weak points of the system, as well as give directions for further development.